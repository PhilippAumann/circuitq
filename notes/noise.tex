\documentclass[]{article} 

\usepackage{hyperref} 
\usepackage{amsmath}
\usepackage{amssymb}
\usepackage[section]{placeins}

% Dirac-Notation
\newcommand{\ket}[1]{| #1 \rangle}
\newcommand{\bra}[1]{\langle #1 |}
\newcommand{\braket}[2]{\langle #1 | #2 \rangle}
\newcommand{\ketbra}[2]{| #1 \rangle \langle #2 | }
% Identity
\newcommand{\identity}{1\!\!1}

\title{$T_1$ and $T_2$ time contributions in superconducting circuits}
\date{} % Activate to display a given date or no date (if empty),
% otherwise the current date is printed 
\begin{document}
	\maketitle
The longitudinal and transverse relaxation rates are given by their inverse time scales. Within the Bloch-Redfield description, the transverse relaxation is given by the longitudinal relaxation and dephasing.~\cite{krantz2019}
\begin{align}
\Gamma_1 &= \frac{1}{T_1} \\
\Gamma_2 &= \frac{1}{T_2} =  \frac{\Gamma_1}{2} + \Gamma_{\varphi}  
\end{align}
Strictly speaking, due to 1/$f$ noise, the decay functions of the off-diagonal terms are non-exponential and the system does not fall into the Bloch-Redfield regime. However for $T_{\varphi}   \gtrsim T_1$, the decay can be approximated by an exponential function. If there is no pure dephasing, it holds that $T_2 = 2T_1$, otherwise we have $T_2 \leq 2 T_1$. 
In the following we give an overview of noise contributions for superconducting qubits, which we could implement in CircuitQ.
\section{$T_1$}

\begin{table}[h]
\hspace*{-2cm}
\centering
\begin{tabular}{|p{2cm}|p{5.7cm}|p{7.5cm}|}
	\hline 
Contribution	& Formula & Remarks \\ 
\hline \hline 
Spontaneous emission & $T_1 = \frac{12 \pi \epsilon_0 \hbar c^3}{d} \frac{1}{\omega_q^3}$~\cite{koch2007} & $d=2eL$: Dipole moment with $L\sim 15$ $\mu$m~\cite{koch2007} \newline $\omega_q$: Qubit frequency \newline
For transmon, contribution to $T_1=0.3$ ms~\cite{koch2007} \\
\hline
%Quasiparticle tunneling (Transmon paper) & In transmon regime~\cite{koch2007} \newline $T_1 = \left( \Gamma_{qp} N_{qp} \sqrt{\frac{k_b T}{\hbar \omega_q}} | \langle g, n_g \pm \frac{1}{2} | e, n_g \rangle |^2 \right)^{-1}$ \newline
%with number of quasiparticles: \newline $N_{qp} = 1 +N_e e^{-\Delta/k_BT}\frac{3\sqrt{2\pi} \sqrt{\Delta k_b T}}{2E_F}$
% & $\Gamma_{qp} =\delta g_T  / 4 \Pi \hbar $: quasiparticle tunneling rate  \newline
%$\delta = 1  /  \nu V$: mean level spacing of reservoir \newline
%$\nu = 3n/2E_F$: density of states \newline
%$n = 18.1 \cdot 10^{22}$~cm${}^{-3}$: conduction electron density \newline
%$E_F=11.7$~eV: for aluminum (as $n$) \newline
%$V=150~\mu$m${}^3$: metal volume \newline
%$g_T=1e^2/h$: junction conductance \newline
%$\Delta=90$~GHz: superconducting gap~\cite{delta} \newline
%$N_e=nV$: number of conduction elecctrons \newline
%$T=20$~mK: temperature  \newline
%For transmon, contribution to  $T_1 \sim 1$ s~\cite{koch2007} \\
%\hline
Quasiparticle tunneling & $T_1 = \left( \sum_j \left| \bra{g} \sin ( \hat{\varphi}_j/2) \ket{e} \right|^{2} S_{qp, j} (\omega_q) \right) ^{-1} $~\cite{catelani2011}& $\hat{\varphi}_j = (\hat{\Phi}_{j,1}-\hat{\Phi}_{j,2}) / \Phi_0$: phase operator \newline
$S_{qp, j} (\omega_q) = x_{qp}\frac{E_{J, j}}{h}\sqrt{\frac{8 \Delta}{\omega_q}}$~\cite{yan2016} \newline 
$E_{J,j}:$ Josephson energy of the $j$-th junction \newline
$x_{qp}=\sqrt{2\pi k_B T/ \Delta}e^{-\Delta / k_B T}$: Quasiparticle density normalized by the density of Cooper pairs~\cite{catelani2011PRL} \newline
$\Delta\approx 1.76 \cdot k_B T_c$: superconducting gap~\cite{gap} \newline
$T_c \approx 1.2$ K: critical temperature in aluminum~\cite{T_c} \\
\hline 
Charge noise & $T_1 = \frac{\hbar^2}{S_{Q}(\omega_q)} \left| \bra{e} \hat{V} \ket{g} \right|^{-2}$~\cite{yan2016}
& $\hat{V}_i=\left(\hat{Q}C^{-1} \right)_i$: voltage operator \newline 
$\hat{Q}$: charge operator vector with the charge-variables on the rows.\newline
$S_{Q}(\omega_q)=A_Q^2\left( \frac{2\pi \cdot 1 Hz}{\omega_q} \right)^{\gamma_Q}$: noise spectral density with $A_Q^2=(10^{-3}e)^2/$Hz~\cite{krantz2019} \\
\hline
Flux noise & $T_1 = \frac{\hbar^2}{S_{\Phi}(\omega_q)} \left| \bra{e} \hat{I} \ket{g} \right|^{-2}$~\cite{yan2016}
& $\hat{I}_i$: current operator, which includes the sum of all $\frac{\Phi_i-\Phi_j}{L_{ij}}$ and $I_{C, ij}  \sin \frac{\Phi_i-\Phi_j}{\Phi_0}$ terms in the circuit that correspond to a distinct loop edge. \newline 
$I_{C, ij} = 2 \pi E_{J, ij} / \Phi_0$: critical current \newline
$S_{\Phi}(\omega_q)=A_{\Phi}^2\left( \frac{2\pi \cdot 1 Hz}{\omega_q} \right)^{\gamma_{\Phi}}$: noise spectral density with $A_{\Phi}^2\approx(10^{-6} \Phi_0)^2/$Hz~\cite{krantz2019} \\
\hline
Purcell effect & We don't include the purcell effect as we exclude external control and readout circuitry from our analysis. 
& For transmon, contribution to  $T_1 \sim 16$ $\mu$s~\cite{koch2007}  \\
\hline

\end{tabular} 
\caption{Overview of the noise contributions to $T_1$.}
\label{tab:T_1}
\end{table}


\section{$T_2$}
The pure dephasing contributions listed in table~\ref{tab:T_2} should vanish at the sweet spot. External parameters are used to perform gates, prepare states or tune the qubit. These parameters are used to manipulate the qubit and are no fundamental components. Therefore we do not consider these contributions for now.

\begin{table}[h]
\hspace*{-2cm}
\centering
\begin{tabular}{|p{2cm}|p{5.5cm}|p{7.5cm}|}
	\hline 
Contribution	& Formula & Remarks \\ 
	\hline \hline 
Offset charge noise (pure dephasing)& $T_2 \sim \frac{\hbar}{A_q} \left| \frac{\partial E_q}{\partial q_{off}} \right|^{-1} $~\cite{koch2007} & This formula can be used for small fluctuations and small frequencies \newline
 $A_q = 10^{-4}e - 10^{-3}e$: noise amplitude \newline
 $E_q$: qubit energy \\
\hline 
Offset flux noise (pure dephasing)& $T_2 \sim \frac{\hbar}{A_{\phi}} \left| \frac{\partial E_q}{\partial \phi_{off}} \right|^{-1} $~\cite{koch2007} & This formula can be used for small fluctuations and small frequencies \newline
 $A_{\phi}= 10^{-6}\Phi_0 - 10^{-5}\Phi_0$: noise amplitude \\
\hline 
\end{tabular} 
\caption{Overview of the noise contributions to $T_2$.}
\label{tab:T_2}
\end{table}


\bibliography{references}
\bibliographystyle{plainurl}

\end{document}
